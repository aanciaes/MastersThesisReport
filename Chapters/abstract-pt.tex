%!TEX root = ../template.tex
%%%%%%%%%%%%%%%%%%%%%%%%%%%%%%%%%%%%%%%%%%%%%%%%%%%%%%%%%%%%%%%%%%%%
%% abstrac-pt.tex
%% NOVA thesis document file
%%
%% Abstract in Portuguese
%%%%%%%%%%%%%%%%%%%%%%%%%%%%%%%%%%%%%%%%%%%%%%%%%%%%%%%%%%%%%%%%%%%%
Os recentes avanços de ambientes de execução confiáveis baseados em hardware fornecem execução isolada, protegida contra sistemas operativos não confiáveis, permitindo o estabelecimento de componentes base de computação de confiança protegidos por hardware. Como o processador fornece esses ambientes de execução confiável e "protegida" (TEE), o seu uso permitirá que os utilizadores executem aplicações com segurança, por exemplo em servidores \textit{cloud} remotos, cujos sistemas operativos e hardware estão expostos a atacantes potencialmente maliciosos assim como administradores de sistema não controlados.
Por outro lado, os \textit{containers} Linux geridos por sistemas \textit{Docker} ou \textit{Kubernetes} são soluções interessantes para poupar recursos físicos, obter tempos de inicialização mais rápidos e flexíveis e maior desempenho de I/O (interfaces de entrada e saída), em comparação com as tradicionais máquinas virtuais (\textit{VM}) activadas pelos hipervisores. No entanto, essas soluções sofrem com software e mecanismos de kernel mais fáceis de comprometerem os dados das aplicações na sua integridade e privacidade.

Esta dissertação projectará, implementará e avaliará um Sistema de  Armazenamento de Dados em Memória Confiável e Focado na Privacidade, utilizando uma biblioteca conteinerizada e protegida por hardware para suportar as suas suposições de capacidade de confiança. Para oferecer suporte para grandes conjuntos de dados, exigindo assim que os dados sejam mapeados fora dos \textit{containers} seguros pelo hardware, a solução planeada utilizará encriptação homomórfica parcial, permitindo que operações executadas no ambiente de execução protegido façam gestão de dados na memória que estão permanentemente cifrados, estando eles mapeados dentro ou fora dos \textit{containers} seguros.

% Palavras-chave do resumo em Português
\begin{keywords}
Segurança de Hardware; Armazenamento de Estrutura de Dados em Memória Confiável e Focado na Privacidade; Encriptação Homomórfica, Ambientes Isolados; Computação Segura; Computação em \textit{Cloud}; Virtualização, Containerização; Disponibilidade; Confiabilidade.
\end{keywords} 
% to add an extra black line