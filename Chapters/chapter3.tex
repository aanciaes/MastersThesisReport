%!TEX root = ../template.tex

\chapter{System Model and Design Options}
\label{cha:system_model_and_design_options}

In this chapter we provide an overview of the system model and implemented architecture along with all of its components. It is explained how all component interact with each other and each component purpose and how they work.

It is also explained how the different security features are implemented into the system in order to provide a secure overall system and achieve the contributions planned for this thesis.

Section \ref{sec:refinement_of_objectives_and_contributions} provides a small recap and refinement of the objectives and contributions of this thesis and the used \gls{TEE} technology.

In section \ref{sec:threat_model_and_security_properties} we describe the basic security assumptions need for our system to work securely. The threat model is a very important part of any security related project as it provides a clear overview of what attacks the system is able to protect against, presents the trustability chain and \gls{TCB} as well as what components and  security properties are out of scoped and not addressed in the project.
Sections \ref{sec:system_model} and \ref{sec:system_architecture} presents an high level view of the system model and implemented architecture of the system. Then, in section \ref{sec:supported_operations} and \ref{sec:operation_flow} we expose the application specific operations supported both key value store specific operations and custom operations implemented by the proxy and also an interaction flow to represent a user interaction with the system.

Finally, as always, section \ref{sec:chapter3_summary} we summarise all the findings and provide a clear transition into chapter \ref{cha:elaboration_plan}, which will present all implementation specific details performed to achieve the described system.

\section{Refinement of Objectives and Contributions} % (fold)
\label{sec:refinement_of_objectives_and_contributions}

The goal of this dissertation as explained on chapter \ref{cha:introduction}, is the design, development and validation with experimental evaluation of a secure in-memory storage (based on a "key-value" model), supported by a hardware-enabled trust computing base.

Regarding the security assumptions, the solution will provide: (i) hardware-isolated in-memory processing engine, designed as a hardware-isolated container facility, enclaved within the Intel-SGX protection guarantees; (ii) hardware-isolated communication endpoints for client access, providing TLS tunnelling with strong TLS 1.3 endpoint encryption parameterisations and support for mutual client/server authentication, and (iii) privacy-enhanced operations to be directly processed on encrypted data sets in memory. The former facility is particularly interesting to combine the possibility to manage protected memory for small data sets and also searchable encrypted data sets that are far larger than the protected memory limits imposed by the SGX memory mapping facility. Furthermore, the solution will target main data structures commonly use fine-grained data items that can include pointers, complex composite types and keys, which do not match well with the coarse-grained paging of the SGX memory extension technique.

In the next subsections we will align the implementation ideas starting by refining the circumvention of limitations in SGX and the threat model assumptions to address our solution.

\subsection{\gls{SGX} Limitations Refinement}
\label{ssec:sgx_limitations_refinement}

Section \ref{ssec:circumvention_of_sgx_limitations} describes in detail the limitations of \gls{SGX}. In our approach we will intend to design a solution that can be leveraged from conventional reference KVS technology, giving the possibility to manage small datasets but also larger datasets (directly mapped in non-protected memory). To circumvent the problem our solution must combine the possibility to use the internal capabilities native to SGX with the possibility of supporting operations managing datasets encrypted in memory, with such operations executed directly over encrypted data. This facility will be provided by the use of partial homomorphic encryption constructions, with data initially encrypted and submitted to the key-value-store solution with cryptographic keys only managed in the client side. Our solution will be designed in order to be possible the support for fine-grained key-value encryption, driven form the application requirements. Our target is the support of a variety of operations provided in a typical KVS API, taking REDIS as the reference solution and in order to support a considerable number of queries currently used by many REDIS-supported applications.

\section{Threat Model and Security Properties}
\label{sec:threat_model_and_security_properties}

The threat model and security properties definition describe the conditions of how we define a secure system. However, to ensure a basic secure system, we must achieve a few key goals and objectives:
 
\textbf{Data Privacy} - Data must remain private to its owner.

\textbf{Data Integrity} - Data must not be compromised, modified or corrupted.

\textbf{Authenticity} - Data and system interactions should be authentic and not spoofed by unauthorised users.

Subsections \ref{ssec:adversarial_model_definition}, \ref{ssec:other_system_assumptions} and \ref{ssec:contermeaseures_for_privacy_preservation} explain under which conditions and assumptions the system will achieve these three main parameters and implement a secure system.

\subsection{Adversarial Model Definition}
\label{ssec:adversarial_model_definition}
	
As the baseline, our threat model will lie on the protection overview stated in SGX’s paper \cite{sgx:7}: \textit{SGX prevents all other software from accessing the code and data located inside an enclave including system software and access from other enclaves. Attempts to modify an enclave’s contents are detected and either prevented or execution is aborted}, which falls in the following adversary model:

\textbf{Isolation by trusted containerisation from malicious code}: The system performs and protects its data from an attacker capable of compromising the system through another application installed on the same system or malicious code existent or injected in the OS or OS hypervisor layers;

\textbf{Privacy protection against insider "Honest but Curious" System Administrators}: The system must be able to protect from an attacker with root access to the machine, with permissions to access and monitor memory-mapped data. This is relevant because we will target our solution as a candidate solution to protect data privacy in a cloud-based key value store solution as a service, preventing data-leakage vulnerabilities exploitable by insider incorrect users or system-administrators.

\textbf{Network Attacks:} All communication to and from the system (supporting client-service operations) should be secure, using proper strong cryptographic parameterisations for TLS 1.3, mutual authenticated handshakes and TLS endpoint executions isolated in SGX-enabled TLS tunnels in communication containers, avoiding attacks against the authentication of the service endpoints, as well as, attacks against the integrity and confidentiality of data flows supporting \gls{REST}/\gls{HTTPS} operations.

\textbf{File system and memory access attacks:} All sensitive data residing outside protected memory should be encrypted and operated in the encrypted form. An attacker can access the physical disks and hardware without the sensitive data being exposed.
	
\subsection{System Assumptions}
\label{ssec:other_system_assumptions}

With the above security baseline considered for the threat model assumptions, the solution must be  resilient to malicious privileged attacks and certain physical attacks. With a controlled and reduced hardware-shielded trust computing base, we want to design a solution that does not rely on the security of operating system managed by cloud providers. Furthermore, our solution must be also resilient to direct conventional physical attacks, such as cold boot attacks, which attempt to retain the DRAM data by freezing the memory chip or even bus probing to sense and to read exposed memory channel between the processor and memory chips. The only weaknesses not covered in our concerns with be the SGX  lack of protection for side-channel attacks.

The system planned has certain assumptions and aspects that are considered to be out of scope for this dissertation:

\begin{itemize}
	\item \textbf{Trusted Client} - The client side is assumed to be completely trusted and correct.
	\item \textbf{\gls{DoS} and \gls{DDoS}} attacks are out of scope.
	\item \textbf{Side Channel Attacks} - It is out of scope any side channel attacks or any related attack not present in \gls{SGX}'s threat model.
	\item \textbf{Physical and Hardware attacks} exploring the \gls{SGX} processing model and its isolation guarantees are out of scope, namely those presented above initially addressed in chapter \ref{cha:related_work}, section \ref{sec:trusted_computing _environments}.
\end{itemize}

\subsection{Countermeasures for Privacy-Preservation}
\label{ssec:contermeaseures_for_privacy_preservation}

Under the described threat model and system assumptions, the system has some measures to achieved the desired security and trustability level.

All instances of the datastore are running in a containerised solution which means that each container is not only isolated from the host but they are also isolated from each other. Data is kept in memory at all times unless persistent disk storage is turned on. Data in main memory is secure when running both in protected and unprotected mode. 

Privacy, integrity and authenticity when running in protected mode are always ensured by the \gls{SGX}'s technology under their threat model and assumptions. But when the system is running in unprotected memory, outside the trusted execution environment, the system will always keep data always encrypted with strong and standard state of the art cryptography algorithms, therefor preserving privacy of data. 

For data integrity preservation, all values are appended with standard checksums calculations and integrity check algorithms. Authenticity is preserved by performing strong and standard cryptographic signing algorithms.

As for the application security, the system provides a secured \gls{API} that only allows authenticated requests and it contains a role based segmentation authorisation with secure and strong passwords and cryptographic keys. This system also applies the principle of least privilege \cite{polp:1} to all actions, where users never have more privileges that they require.

Privacy is also preserved on all communications with the use of the strongest transport layer security algorithms currently available, and established trust between all components with a trusted certificate chain.

\section{System Model} % (fold)
\label{sec:system_model}

The main goal of this project is modelled in figure \ref{fig:system_model_overview}. As shown, the model can be divided in four main components - the \textbf{client}, the \textbf{proxy server}, the \textbf{key-value storage server} and the \textbf{authentication server}.

% \subsection{General Overview}
% \label{ssec:general_overview}

\begin{figure}[htbp]
  \centering{\includegraphics[width=1.0\linewidth]{FinalSystemModel}}%
  \caption{System Model Overview}
  \label{fig:system_model_overview}
\end{figure}

The system complies with the adversary and threat model explained in section \ref{ssec:adversarial_model_definition}. The user is not aware of any implementation details and has seamless interaction with the system despite the architecture and implementation provided by the backend, meaning that all solution expose the same \gls{API} and support an interface as equal as possible to the unprotected version of the key-value store. 

All of the four main components will be explained in the subsections below.

\subsection{Key-Value Storage Server}
\label{ssec:key-value_storage_server}

The storage server is a key-value store meant to hold data required by the user. This data is kept in memory so it allows fast read, writes, updates and deletes.

In this project, there will be two different kinds of storage servers. The \textbf{secured} and \textbf{unsecure} storage server. This nomenclature do not reflect the privacy, integrity and authenticity security problems as they are preserved on both components but it reflects the environment which they will run on.

The \textbf{secure} storage runs on a trusted execution environment (trusted hardware) protected by a secure physical processor present on the host machine. This processor provides the security features that allow for memory to be operated in plain text without breaking privacy or data security as explained in section \ref{sec:trusted_computing _environments}.

As for the \textbf{unsecure} storage, it describes a storage service that runs on unprotected memory regions and on untrusted hardware. This means that data must be actively secure by encryption, integrity checks and authentication features in order to provide and preserve privacy.

We must mention that the above system model addresses the replication facilities as leveraged from the Redis solution and tries to offer the same availability conditions as the original Redis service. Both secure and unsecure configurations will provide availability by running replicated Redis instances in a cluster. This availability provided by the replication of Redis instances that can run as cloud service instances (that can be distributed among different cloud providers and geo-replicated datacenters) only offers consistency guarantees under a fail-stop model (not extended to byzantine fault tolerance or byzantine intrusion tolerance, that are conditions out of scope of our planned dissertation).

Every Redis instance running on a secure configuration will also provide an API for remote attestation, so their hardware stack and software can be attested by remote parties, to provide trust to the users.

\subsection{Proxy Server}
\label{ssec:system_model_proxy_server}

The proxy server serves as a gateway to the storage instances. It is a single point of entry to the system which as some advantages. Not only it provides an abstraction to a backend which can be replicated across multiple instances across multiple geo-locations, it also can add additional features to the system. 

With a centralised gateway we have a centralised access management with authentication and authorisation features. Not only that, but the connection to multiple instances of secure data storages with different authentication mechanisms and secrets can also be centralised in one place, which allows the user to have just a single username/password pair to connect to all the storage instances.

The proxy server manages security properties when connecting to instances on a unsecure storage configuration. It is the proxy that handles security features like encryption and decryption, integrity and authentication checks, and provides a completely encrypted and private storage service.

It also exports a \gls{REST} \gls{API} which can be used by multiple clients with different implementations and also offers custom features that the standard data storage system does not. Custom operations will be explained below, on section \ref{sec:supported_operations}.

The proxy server serves as a single point of access for the multiple storage instances remote attestation features. It attests every instance and feedback the user on the hardware stack and software state of each instance. Furthermore, the proxy server itself runs on a trusted execution environment leveraging secure trusted hardware and can also be attested by the clients.

\subsection{Authentication Server}
\label{ssec:authentication_server}

An external authentication server is required to centralised user management. It is responsible for user authentication and verification. A user authenticates against the authentication server receiving an access token, that it provides to the proxy on every request. The proxy can verify the token and check if it is valid and what permissions this user has, and can authorise or reject the access to a particular endpoint or system functionality.

An external authentication server relieves the proxy of user authentication and management, and outsources it to a standardised open source system that implements security standards of identity and access management. This is also import so that the proxy server can, if needed, be replicated.

\subsection{Client}
\label{ssec:client}

The client, or user, is the one that will consume the exposed \glspl{API} by the proxy server. In the case of this project, the client will be a representative benchmark tester that will use the exposed endpoints to record benchmark times and various relevant evaluation criteria so all storage configurations and replication mechanism can be compared.

Although just a simulated client/tester, the \glspl{API} are consumed the same way a real user would consumed them, so, benchmarks are a representation of real system usage.

\section{System Architecture}
\label{sec:system_architecture}

Figure \ref{fig:system_architecture_stack} describes the hardware and software representative stack of the infrastructure \footnote{The described stack is a representative stack of the machine layers and it does not represent a true overhead ladder}. Figure \ref{fig:storage_stack} shows how the storage service are deployed onto the cloud provider's machines. The machines hardware provide a very specific and physical processor that implements a trusted execution environment. Running on the operating system, a containerisation solution runs the \gls{TEE} virtualisation framework in order for the container running both the key value stores and the attestation services being able to access the \gls{TEE}. On the right hand side of this figure, we can see a storage service that runs with no \gls{TEE} virtualisation system, and that is the deployment of an unprotected key value store configuration.

On figure \ref{fig:proxt_stack} the stack provided describes a system also running in the confinements of a trusted execution environment and therefor, also providing an attestation service.

\begin{figure}[htbp]
  \centering
  \subcaptionbox{Storage Stack\label{fig:storage_stack}}%
    {\includegraphics[width=0.35\linewidth]{storage_architecture}}%
    \hspace{5em}
  \subcaptionbox{Proxy Stack\label{fig:proxt_stack}}%
    {\includegraphics[width=0.35\linewidth]{proxy_architecture}}%
  \caption{System Architecture Stack}
  \label{fig:system_architecture_stack}
\end{figure}

\section{Supported Operations}
\label{sec:supported_operations}

The exposed \gls{API} can perform some operations over the storage system. Some operations are out of the box key-value stores operations, and some were implemented in the proxy so to give a customised and more complex operations.

The next subsections will iterate and explain the supported operations, and we need to indicate that all operations listed can work on both main data storage configurations (protected and unprotected), where operations are performed whether data is maintained on clear text, securely inside a trusted execution environment or maintained encrypted in unprotected memory regions.

\subsection{Role-Based Authorisation}
\label{ssec:role-based_authorisation}

The system by default rejects all unauthenticated requests, which means, users do need to be registered and have credentials to access the system.

As explained before, user management is performed by the external authentication server, and each user has a role assigned. Different roles can do different actions on the system and role based authorisation can be completely customised.

In this thesis we operate with the principle of least privileges, and came up with two different roles: a \textbf{BasicUser} which can only perform read operations and cannot alter the system state (basically a readonly user), and an \textbf{Administrator} which can perform all actions on the system. These roles are just a representation of the role based authorisation that the system performs before accepting a request into the server.

\subsection{Key-Value Storage Operations}
\label{ssec:key-value_storage_operations}

The system is ready to perform a set of operations from the key value storage of choosing, without modification. However, this is not a full fledge solution, and accommodations had to be made for the tow main configurations of the storage system: protected and unprotected.

The supported operations are as follow:

\begin{itemize}
  \item \textbf{Set} - Stores a value into the database associated with a key.
  \item \textbf{Get} - Retrieves a value associated to the provided key.
  \item \textbf{Set on List} - Creates a list of values associated to a key.
  \item \textbf{Get List} - Retrieves all values associated with the given key.
  \item \textbf{Set on List with Score} - Creates a list of values associated to a key. Each value has associated a score (integer)
  \item \textbf{Get on List between Score} - Retrieves all values associated with the given key with scores between provided scores. 
\end{itemize}

\subsection{Proxy Enabled Operations}
\label{ssec:proxy_enabled_operations}

The proxy server implementation allows for the implementation of another set of operations not supported out of the box from the key value store:

\textbf{Sum}, given a key and a number, the server can fetch the value associated with given key and add it to the number provided.

\textbf{Subtraction}, given a key and a number, the server can fetch the value associated with given key and subtract it to the number provided.

\textbf{Multiplication}, given a key and a number, the server can fetch the value associated with given key and multiply it to the number provided.

\textbf{Search on List}, where the server takes a search term, match it against each value of a provided key of a list and return only the matching values.

\subsection{Attestation}
\label{ssec:attestation}

Users can also request the remote attestation of the complete system. The proxy server is responsible to contact each storage server's attestation service and request attestation quotes. Each quote is them gathered in the proxy, which attests it self and returned to the client. 

The correctness of the system is then determined by the client which analises the quotes provided by each system, and decides whether  or not the system is correct and protected and can continue on using it.

% TODO: Write about CAS and LAS. Write about Redis TLS attestation and AIK attestation service attestation

\section{Operation Flow}
\label{sec:operation_flow}

Figure \ref{fig:operation_flow} shows an example of a flow that can occur between all system components.

\begin{figure}[htbp]
  \centering{\includegraphics[width=1.0\linewidth]{UserFlow}}%
  \caption{Operation Flow}
  \label{fig:operation_flow}
\end{figure}

Following the flow, we can see that the first interaction is with the external authentication server by providing the user credentials (\textbf{1}) and (if login successful) receiving an access token in return (\textbf{2}). 

Then, the user uses the access token to attest the system and make sure that hardware and software stacks are working as expected. It requests an attestation to the proxy (\textbf{3}) with the access token retrieved on login. The proxy checks the access token with the authentication server (\textbf{4}) and verifies if it is correct and has the necessary roles to access the attestation endpoint. If the verification fails, a proxy returns an authentication/authorisation fail back to the client (\textbf{5}). 

If it succeeds, the attestation can proceed and the proxy server contact all storage server attestation services and gathers all quotes (\textbf{6}), and while it's doing that, also starts an self attestation process (\textbf{7}) and returns all quotes to the client. 

Now, the client must analise the quotes from the attestation and decide whether or not it trusts the system to continue (\textbf{8}). If it deems the system incorrect, it fails but if the client trusts the system, it can carry on with normal operations to the proxy and to the storage server (\textbf{9}).

\section{Summary} % (fold)
\label{sec:chapter3_summary}

To achieve the objectives and contributions of this thesis, the system should rely on a physical hardware processor that can implement an isolated execution environment, to protect the system against insider threats, operating system and hypervisors vulnerabilities, and several other attacks already explained in the adversary and threat model of this project. That processor, as shown on system architecture section is present in the used machines, and to allow for a light weight, easily portable and deployable system, an TEE virtualisation technology was used in order to a containerised application can take advantage of the physical processor.

However this technology still has some limitations, and to address a memory limitations, the system also allows for a storage system to be placed in unprotected memory and still maintain security, privacy and integrity properties.

This way, we can maintain a system, running an unmodified key value storage application and deploy it safely on the cloud, or any other machine provider and be sure that data is as protected as it can be.

Not only that, but all the hardware stack and software can be attested so that, on a unlikely event of attack, data corruption or leaked vulnerability occurs, the user can always, to a high percentage of trust, define weather or not the underlying system is correct and can be trusted.