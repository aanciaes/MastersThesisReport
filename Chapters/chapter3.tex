%!TEX root = ../template.tex
%%%%%%%%%%%%%%%%%%%%%%%%%%%%%%%%%%%%%%%%%%%%%%%%%%%%%%%%%%%%%%%%%%%%
%% chapter3.tex
%% NOVA thesis document file
%%
%% Chapter with a short laext tutorial and examples
%%%%%%%%%%%%%%%%%%%%%%%%%%%%%%%%%%%%%%%%%%%%%%%%%%%%%%%%%%%%%%%%%%%%
\chapter{3.	Approach to Elaboration Phase}
\label{cha:approach_to_elaboration_phase}

This Chapter aims at exemplifying how to do common stuff with \LaTeX. We also show some stuff which is not that common! ;)

Please, use these examples as a starting point, but you should always consider using the \emph{Big Oracle} (aka, \href{http://www.google.com}{Google}, your best friend) to search for additional information or alternative ways for achieving similar results.

\section{Refinement of Objectives and Contributions} % (fold)
\label{sec:refinement_of_objectives_and_contributions}

Refinement of objectives and contributions

% section document_structure (end)


\section{System Model Approach} % (fold)
\label{sec:system_model_approach}

System model approach

% section dealing_with_bibliogrpahy (end)


\section{Planned Architecture and Implementation} % (fold)
\label{sec:planned_architecture_and_implementation}

Planned architecture and implementation

% section inserting_tables (end)


\section{Planned Testbench Environments} % (fold)
\label{sec:planned_testbench_environments}

Planned testbench environments

% section importing_images (end)


\section{Relevant Evaluation Criteria} % (fold)
\label{sec:relevant_evaluation_criteria}

% \subsection{Inserting Figures Wrapped with text} % (fold)
% \label{ssec:inserting_images_wrapped_with_text}
% 
% You should only use this feature is \emph{really} necessary. This means, you have a very small image, that will look lonely just with text above and below.
% 
% In this case, you must use the \verb!wrapfiure! package.  To use \verb!wrapfig!, you must first add this to the preamble:
% 
% \begin{wrapfigure}{l}{2.5cm}
%   \centering
%     \includegraphics[width=2cm]{snowman-vectorial}
%   \caption{A snow-man}
% \end{wrapfigure}	
% 
% \noindent\verb!\usepackage{wrapfig}!\\
% This then gives you access to:\\
% \verb!\begin{wrapfigure}[lineheight]{alignment}{width}!\\
% Alignment can normally be either ``l'' for left, or ``r'' for right. Lowercase ``l'' or ``r'' forces the figure to start precisely where specified (and may cause it to run over page breaks), while capital ``L'' or ``R'' allows the figure to float. If you defined your document as twosided, the alignment can also be ``i'' for inside or ``o'' for outside, as well as ``I'' or ``O''. The width is obviously the width of the figure. The example above was introduced with:
% \lstset{language=TeX, morekeywords={\begin,\includegraphics,\caption}, caption=Wrapfig Example, label=lst:latex_example}
% \begin{lstlisting}
% 	\begin{wrapfigure}{l}{2.5cm}
% 	  \centering
% 	    \includegraphics[width=2cm]{snowman-vectorial}
% 	  \caption{A snow-man}
% 	\end{wrapfigure}	
% \end{lstlisting}

% subsection inserting_images_wrapped_with_text (end)

% section floats_figures_and_captions (end)

Chapter 3

\begin{figure}[htbp]
  \centering
  \subcaptionbox{One sub-figure\label{fig:leftsubfig}}%
    {\includegraphics[width=0.5\linewidth]{knitting-vectorial}}%
  \subcaptionbox{Another sub-figure\label{fig:rightsubfig}}%
    {\includegraphics[width=0.5\linewidth]{knitting-vectorial}}%
  \caption{A figure with two sub-figures!}
  \label{fig:fig2subfig}
\end{figure}

\textbf{And this is a small text that references the Figure~\ref{fig:fig2subfig} and its Subfigures~\ref{fig:leftsubfig} and~\ref{fig:rightsubfig}.}

Chapter 3