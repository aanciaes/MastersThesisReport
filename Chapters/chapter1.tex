%!TEX root = ../template.tex
%%%%%%%%%%%%%%%%%%%%%%%%%%%%%%%%%%%%%%%%%%%%%%%%%%%%%%%%%%%%%%%%%%%
%% chapter1.tex
%% NOVA thesis document file
%%
%% Chapter with introduciton
%%%%%%%%%%%%%%%%%%%%%%%%%%%%%%%%%%%%%%%%%%%%%%%%%%%%%%%%%%%%%%%%%%%
\newcommand{\novathesis}{\emph{novathesis}}
\newcommand{\novathesisclass}{\texttt{novathesis.cls}}


\chapter{Introduction}
\label{cha:introduction}

In this chapter it's presented the context and motivation for this thesis, the main problem statement followed by the goals and objectives and all the planned contributions. In the end, it is presented the structure used in the following chapters of the document.

\section{Context and Motivation} % (fold)
\label{sec:context_and_motivation}

Cloud computing has gone through many steps that include grid and utility computing, application service provision and software as a service before reaching the level we know these days. The concept of delivering continuous resources through a global network is rooted in the 1960's. Some experts credit the professor and computer scientist John McCarthy \cite{john_mcCarthy:1} be proposing the concept of computation being delivered as a public utility.

Then, around 1970's the concept of the virtual machine (\gls{VM}) started to gain popularity as it permitted multiple distinct computing environments to reside on one physical environment.

One of the first major cloud computing moments was the arrival of \textit{salesforce.com} that pioneered the concept of delivering enterprise applications via a simple website. Later, around the 2000's, current big names like Oracle, SAP, Google, Amazon and Microsoft joined the trend and made the cloud world as it is today. \cite{cloud_history:1} \cite{cloud_history:2}

Over the past decade, cloud computing has evolve from something service providers told companies they should adopt to becoming the technology heart of not only major companies, but medium sized enterprises, small start-ups, personal projects and pretty much anyone who works in the computer science world. 

Recent studies are foreseeing that 83{\%} of enterprise workloads will be on the cloud by 2020 \cite{cloud_statistic:1}. The array of services provided now are endless and the costs are attractive to businesses. These services allow developers to only pay for resource usage, and to take advantage of all the power of very large companies. Scalability at request, reliability with daily backups and seamless integration with a lot of other services are some advantages of moving to the cloud. And all of these functionalities without having to manage big infrastructures and a lot of servers, networks, disks, etc... \cite{cloud_benefits:1}.

All of this of that data and processing happening in someone else's machine started to raise privacy and security concerns. It has become a very attractive target for malicious hackers to attack cloud providers due to the amount of data they process and hold on their services. The best security researchers are always working with the providers to try and mitigate all bugs and vulnerabilities on their very large platforms which has become also a big attack vector. It has been reported by Microsoft, that \textit{"There was a 300 percent increase in Microsoft cloud-based user accounts attacked year-over-year (Q1-2016 to Q1-2017)."} and \textit{"The number of account sign-ins attempted from malicious IP addresses has increased by 44 percent year over year in Q1-2017."} \cite{cloud_attacks:1}. Another example published on the Washington Post describes a sophisticated Man-in-the-Middle (\gls{MIM}) cyber-attack that has targeted Apple’s iCloud service in China, in an apparent attempt to collect user names, passwords and other private information. Also, Amazon Web Services has been in 2019 hit by a massive \gls{DDoS} (Distributed Denial of Service) attack that kept the system down for about 8 hours straight, which can mean thousands of dollars lost by clients \cite{cloud_attacks:3}.

The well known Edward Snowden scandal \cite{snowden:1}, although not directly related, gave the world another perspective about the the security provided by the cloud providers, that keep the user's data secure from other hackers, but could technically be accessed by the respective provider, or by any system administrator with physical or remote access to the machines.

A previously mentioned study also reflects that \textit{"66{\%} of IT professionals say security is their greatest concern in adopting an enterprise cloud computing strategy"} \cite{cloud_statistic:1}.

% section a_bit_of_history (end)


\section{Objective} % (fold)
\label{sec:objective}

It is up to you, the student, to read the FCT and/or NOVA regulations on how to format and submit your MSc or PhD dissertation.  

This template is endorsed by the FCT-NOVA and even linked from its web pages, but it is not an official template.
%
This template exists to make your life easier, but in the end of the line you are accountable for both the looks and the contents of the document you submit as your dissertation.

\section{Planned Contributions} % (fold)
\label{sec:planned_contributions}

It is up to you, the student, to read the FCT and/or NOVA regulations on how to format and submit your MSc or PhD dissertation.  

This template is endorsed by the FCT-NOVA and even linked from its web pages, but it is not an official template.
%
This template exists to make your life easier, but in the end of the line you are accountable for both the looks and the contents of the document you submit as your dissertation.

\section{Report Organization}
\label{report_organization}

It is up to you, the student, to read the FCT and/or NOVA regulations on how to format and submit your MSc or PhD dissertation.  

This template is endorsed by the FCT-NOVA and even linked from its web pages, but it is not an official template.
%
This template exists to make your life easier, but in the end of the line you are accountable for both the looks and the contents of the document you submit as your dissertation.
