%!TEX root = ../template.tex
%%%%%%%%%%%%%%%%%%%%%%%%%%%%%%%%%%%%%%%%%%%%%%%%%%%%%%%%%%%%%%%%%%%%
%% chapter2.tex
%% NOVA thesis document file
%%
%% Chapter with the template manual
%%%%%%%%%%%%%%%%%%%%%%%%%%%%%%%%%%%%%%%%%%%%%%%%%%%%%%%%%%%%%%%%%%%%
\chapter{Related Work}
\label{cha:related_work}

This chapter presents and briefly discusses the related work and the study performed beforehand in order to guide and give some context to the reader. It will present work that was used as the basis of this thesis, existent technologies and their relation with this project, and some comparisons between those exiting technologies, the problem addressed in this thesis and the solutions proposed to solve, or better address, those very same problems.

First, in section 2.1 we explain and discuss for the first time the definition of a Key-Value Store. We  present some use cases, current technology available, their differences and most importantly their security models and concerns.

Having discussed the software, section 2.2 will then address the environment on where the previously talked software will run, most specifically the hardware. It explains and present the different ways to secure and authenticate the hardware, prevent hardware-based attacks and discuss some of the current products available and how they will be used across this thesis.

Section 2.3 will then make the bridge between software and hardware. It explains how Key-Value stores are currently being run on secure environments. 
This chapter will be focused on the Intel SGX secure model and explain the advantages and disadvantages of this module

To conclude the chapter, section 2.4... 

//TODO: complete

Along the next chapter we summarize the main relevant ideas that can be retained from each section for our objectives and expected goals.

\section{Key-Value Stores} % (fold)
\label{sec:key-value_stores}

Key value stores are the simplest form of what computer scientists call a database. The simplicity lies on associating a value to a certain key and storing that pair, as well as retrieving the values of known keys. \cite{db-engine:1}

\lstset{language=Bash, caption=Redis Set \& Get, label=lst:redisSetGet}
\begin{lstlisting}
redis> SET mykey "Hello"
"OK"
redis> GET mykey
"Hello"
redis> 
\end{lstlisting}

Is this simplicity that makes this technology very attractive to developers. The ease of use, its high performance and speed are key aspects in favour of this technologies. However, simply working with keys and values might not be enough to more complex applications, and that is why Key-Value store product developers are introducing new features in order to make them appealing to a broader mass of users, always keeping them lightweight and fast.

For that lightweight and fast attributes, most of the key-value stores work in the computer memory. This allows fast get and write operations as opposed to persistent disk storage. Although, they work mainly in memory, most of the solutions offer some persistent mechanism so we can make use of its performance but still persist data in case of a disaster, server failure or any crash.

\gls{KVS}s have been evolving for years and some are now more than a single key-value store module. A lot of them are now supporting a multi-model storage. That means that a value can be more than a single integer or a string. For example, Redis \cite{redis:1} as a multi-model store is not only a key-value store, but also \cite{redis:2}:

\begin{itemize}
	\item Document Store - \textit{"nonrelational database that is designed to store and query data as JSON-like documents"} \cite{aws-nosql:1}
	\item Graph \gls{DBMS} - \textit{"Graph databases are purpose-built to store and navigate relationships. Use nodes to store data entities, and edges to store relationships between entities"} \cite{aws-nosql:2} 
	\item Search Engine - \textit{"nonrelational database that is dedicated to the search of data content. Use indexes to categorize the similar characteristics among data"} \cite{aws-nosql:3} 
	\item Time Series \gls{DBMS} - \textit{"Provides optimum support for working with time-dependent data. Each entry has a timestamp, the data arrives in time order and time represents a primary axis for the information."} \cite{timeSeries:1}
\end{itemize}

So, the \gls{KVS} world is becoming more and more versatile as the years pass.

In the next subsections its discussed and presented the overview of the current \gls{KVS} technology. We picked the some top KVSs technologies nowadays according to db-engines \cite{db-engine:2}.

\subsection{Memcached} % (fold)
\label{ssec:memcached}

Memcached \cite{memcached:1} is a free and open source key-value store released in 2003. It is described as a high performance distributed memory object caching system.

It is design to hold small chunks of data (strings and objects) to work as a cache for results of database calls, API calls, or page rendering. Its biggest use case is for use in speeding up dynamic web applications by alleviating database load.

This system lies on the simpler key-value store spectrum. It takes advantages of the simplicity of a key-value store to edge ease of development, and solving many problems facing large data caches. Its API is available for most popular languages. It has a \gls{LRU} eviction technique which means that items will expire a specified amount of time. 

When it comes to system availability and reliability, Memcached has an interesting approach. In order to keep it blazing fast, there is no communication between server instances in a cluster. Memcached servers are unaware of each other. There is no crosstalk, no synchronization, no broadcasting, no replication. Adding servers will only increase the available memory.

As for its security context, Memcached spends very little, if any, effort in securing the systems for random internet connections. The servers only have support for SASL \cite{sasl:1} authentication mechanism. This method of authentication is not implemented as end-to-end encryption, it only provides restriction access to the daemon, but it does not hide communications over the network. That means it is not meant to be exposed to the internet or to any untrusted users \cite{memcached:2}.

\subsection{Redis} % (fold)
\label{ssec:redis}

Redis \cite{redis:1} is an in-memory data structure store that can be used as a database, cache and also a message broker. Redis focuses on performance, so most of its decisions prioritize high performance and very low latency.

It has been benchmarked as the world's fastest database \cite{redis:3} and together with a their multi-model and its rich set of operations that can be performed over data it has been the leading key-value store according to use and popularity for a multiple set of years \cite{db-engine:2}.

\lstset{language=Bash, caption=How Fast is Redis, label=lst:redisBenchmark}
\begin{lstlisting}
redis-benchmark -t set -r 100000 -n 1000000
====== SET ======
1000000 requests completed in 8.78 seconds
50 parallel clients
3 bytes payload
keep alive: 1

99.59% <= 1 milliseconds
99.98% <= 2 milliseconds
100.00% <= 2 milliseconds
113934.14 requests per second
\end{lstlisting}

As said before, Redis is now not a simple \gls{KVS}. It supports data structures such as strings, hashes, lists, sets, sorted sets with range queries, bitmaps, hyperloglogs, geospatial indexes with radius queries and streams. It also has built-in replication, server side scripting, \gls{LRU} eviction, concept of transactions and different levels of persistence. It provides high availability and automatic partitioning as well.

Security is not Redis' primarily concern (just like others). \textit{"In general, Redis is not optimized for maximum security but for maximum performance and simplicity"} \cite{redis:4}. It is design to be access by trusted clients inside trusted networks. This means that it is not supposed to be publicly exposed. Redis implements a simple authentication system with a password on the configuration file for client authentication.
It is also advised to run it behind a proxy to enable some ACL policies and SSL network security.

There are a few other security concerns that Redis addresses, but has we can now start to see, in this types of stores, security falls behind performance and usability.

\subsection{Amazon Dynamo DB}
\label{ssec:amazon_dynamo_db}

Amazon Dynamo DB

\subsection{Microsoft Azure Cosmos DB}
\label{ssec:microsoft_azure_cosmos_db}

Microsoft Azure Cosmos DB

\subsection{Microsoft Azure Redis Cache}
\label{ssec:microsoft_azure_redis_cache}

Microsoft Azure Redis Cache

\subsection{Aerospike}
\label{ssec:aerospike}

Aerospike

\subsection{Discussion}
\label{ssec:s1_discussion}

Discussion

\section{Trusted  Computing Environments} % (fold)
\label{sec:trusted_computing _environments}

In this section we will provide some additional considerations about some of the customizations available as class options.

\subsection{TPM – Trusted Platform Modules } % (fold)
\label{ssec:trusted_platform_modules}

The choice of the main language with the option “\texttt{lang=OPT}” affects:

\begin{itemize}
	\item \textbf{The order of the summaries.} First is printed the abstract in the main language and then in the foreign language. This means that if your main language for the document in English, you will see first the “abstract” (in English) and then the “resumo” (in Portuguese). If you switch the main language for the document for Portuguese, it will also automatically switch the order of the summaries to “resumo” and then “abstract”.
	\item \textbf{The names for document sectioning.} E.g., ``Chapter'' vs.\ ``Capítulo'', ``Table of Contents'' vs.\ ``Índice'', ``Figure'' vs.\ ``Figura'', etc.
	\item \textbf{The type of documents in the bibliogrpahy.} E.g., ``Technical Report'' vs.\ ``Relatório Técnico'', ``PhD Thesis'' vs.\ ``Tese de Doutoramento'', etc.
\end{itemize} 

No mater which language you chose, you will always have the appropriate hyphenation rules according to the language at that point. You always get Portuguese hyphenation rules in the ``Resumo'', english hyphenation rules in the ``Abstract'', and then the main language hyphenation rules for the rest of the document.

% subsection the_main_language (end).

% section additional_consideration (end)


\subsection{TPM - Enabled Software Attestation} % (fold)
\label{ssec:enabled _software_attestation}

You must choose the class of text for the document. The available options are:

\begin{enumerate}
	\item \textbf{bsc} --- BSc graduation report.
	\item \textbf{*mscplan} --- Preparation of MSc dissertation. This is a preliminary report graduate students at DI-FCT-NOVA must prepare to conclude the first semester of the two-semesters MSc work. The files specified by \verb!\dedicatoryfile! and \verb!\acknowledgmentsfile! are ignored, even if present, for this class of document.
	\item \textbf{msc} --- MSc dissertation.
	\item \textbf{phdprop} ---  Proposal for a PhD work. The files specified by \verb!\dedicatoryfile! and \verb!\acknowledgmentsfile! are ignored, even if present, for this class of document.
	\item \textbf{prepphd} ---  Preparation of a PhD thesis. This is a preliminary report PhD students at DI-FCT-NOVA must prepare before the end of the third semester of PhD work. The files specified by \verb!\dedicatoryfile! and \verb!\acknowledgmentsfile! are ignored, even if present, for this class of document.
	\item \textbf{phd} --- PhD dissertation.
\end{enumerate}
% subsection class_of_text (end)

% ============
% = Printing =
% ============
\subsection{HSM – Hardware Security Modules} % (fold)
\label{ssec:hardware_security_modules}

You must choose how your document will be printed. The available options are:
\begin{enumerate}
	\item \textbf{oneside} --- Single side page printing.
	\item \textbf{*twoside} --- Double sided page printing.
\end{enumerate}
% subsection printing (end)

% =============
% = Font Size =
% =============
\subsection{Trusted Execution Environments} % (fold)
\label{ssec:trusted_execution_environments}

You must select the encoding for your text. The available options are:
\begin{enumerate}
	\item \textbf{11pt} --- Eleven (11) points font size.
	\item \textbf{*12pt} --- Twelve (12) points font size. You should really stick to 12pt\ldots
\end{enumerate}
% subsection font_size (end)

% =================
% = Text encoding =
% =================
\subsection{Intel SGX} % (fold)
\label{ssec:Intel_sgx}

You must choose the font size for your document. The available options are:
\begin{enumerate}
	\item \textbf{latin1} --- Use Latin-1 (\href{http://en.wikipedia.org/wiki/ISO/IEC_8859-1}{ISO 8859-1}) encoding.  Most probably you should use this option if you use Windows;
	\item \textbf{utf8} --- Use \href{http://en.wikipedia.org/wiki/UTF-8}{UTF8} encoding.    Most probably you should use this option if you are not using Windows.
\end{enumerate}
% subsection font_size (end)

% ============
% = Examples =
% ============
\subsection{Sanctum} % (fold)
\label{ssec:sanctum}

Let's have a look at a couple of examples:

\begin{itemize}
	\item Preparation of PhD thesis, in portuguese, with 11pt size and to be printed single sided (I wonder why one would do this!)\\
	\verb!\documentclass[prepphd,pt,11pt,oneside,latin1]{thesisdifct-nova}!
	\item MSc dissertation, in english, with 12pt size and to be printed double sided\\
	\verb!\documentclass[msc,en,12pt,twoside,utf8]{thesisdifct-nova}!
\end{itemize}
% subsection examples (end)

\subsection{ARM Trust Zone} % fold
\label{ssec:arm_trust_zone}

ARM Trust Zone

\subsection{Discussion}
\label{ssec:s2_discussion}

Discussion

\section{TEE/SGX Enabled Key Value Stores} % (fold)
\label{sec:sgx_enabled_key_value_stores}


\subsection{Trusted Execution with Intel SGX}
\label{ssec:trusted_execution_with_sgx}

Trusted Execution with Intel SGX

\subsection{Circumvention of SGX Limitations}
\label{ssec:circumvention_of_sgx_limitations}

Circumvention of SGX Limitations

\subsection{SGX-Enabled Secure Databases}
\label{ssec:sgx_enabled_secure_databases}

SGX-Enabled Secure Databases

\subsubsection{Enclave DB}
\label{sssec:enclave_db}

Enclave DB

\subsubsection{Pesos DB}
\label{sssec:pesos_db}

Pesos DB

\subsubsection{Speicher}
\label{sssec:speicher}

Speicher

\subsubsection{ShieldStore}
\label{sssec:shieldstore}

ShieldStore

\subsection{Discussion}
\label{ssec:s3_discussion}

Discussion

% section how_to_write_using_latex (end)



\section{Related Work Balance and Critical Analysis}
\label{sec:related_work_balance_and_critical_analysis}
%
% \todo[inline]{A a note in a line by itself.}
%
Foo Bar
%
% Please note that
% \begin{center}
%   \textbf{\large this package and template are not official for FCT/NOVA}.
% \end{center}
