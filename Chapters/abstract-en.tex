%!TEX root = ../template.tex
%%%%%%%%%%%%%%%%%%%%%%%%%%%%%%%%%%%%%%%%%%%%%%%%%%%%%%%%%%%%%%%%%%%%
%% abstrac-en.tex
%% NOVA thesis document file
%%
%% Abstract in English
%%%%%%%%%%%%%%%%%%%%%%%%%%%%%%%%%%%%%%%%%%%%%%%%%%%%%%%%%%%%%%%%%%%%
The recent advent of hardware-based trusted execution environments provides isolated execution, protected from untrusted operating systems, allowing for the establishment of hardware-shielded trust computing base components. As the processor provides such “shielded” trusted execution environment (TEE), their use will allow users to run applications securely, for example on the remote cloud servers, whose operating systems and hardware are exposed to potentially malicious remote attackers and non-controlled system administrators’ staff. On the other hand, Linux containers managed by Docker or Kubernetes are interesting solutions to provide lower resource footprints, faster and flexible startup times, and higher I/O performance, compared with virtual machines (VM) enabled by hypervisors. However, these solutions suffer from software kernel mechanisms, easier to be compromised in confidentiality and integrity assumptions of supported application data.
This dissertation will design, implement and evaluate a Trusted and Privacy-Enhanced In Memory Data Store, making use of a hardware-shielded containerised OS-library to support its trust-ability assumptions. To support large datasets, requiring data to be mapped outside those hardware-enabled containers, our targeted solution will use partial homomorphic encryption, allowing trusted operations executed in the protected execution environment to manage in-memory always-encrypted data, that can be or not mapped inside the TEE.

% Palavras-chave do resumo em Inglês
\begin{keywords}
Hardware Security; Privacy-Enhanced Data Store; Homomorphic Encryption; Isolated Environments; Trusted Computing; Cloud Computing; Virtualisation; Containerisation; Availability; Reliability.
\end{keywords} 
