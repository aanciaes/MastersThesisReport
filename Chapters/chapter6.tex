%!TEX root = ../template.tex

\chapter{Conclusions}
\label{cha:conclusions}

This chapter resumes and summarises all the work performed on this thesis and presents final thoughts that were concluded from the related work, system model and architecture, implementation and design options and the evaluations taken from all the tests performed.

Section \ref{sec:main_conclusion} recaptures the objectives and contributions and matches them against the results obtained from the system implementation. Section \ref{sec:open_issues_and_limitations} details the issues and limitations of the system that were discovered during the study of the system and corresponding implementation, and section \ref{sec:future_work_directions} discusses future work that can be performed in order to improve the system and continue the work presented in this thesis, as well as fixing and completing open issues and limitations presented before.

\section{Main Conclusions}
\label{sec:main_conclusion}

Cloud computing adoptions keeps growing with every year that passes and companies and enterprises find more and more attractive some of the features the cloud providers offer such as easy deployment, automatic and seamless scalability as well as replication and pay-as-you-go payment structures and business model. However, data privacy is another trend that as arisen in the past few years, and people, now more that ever, want to hold control of their own data, keep it private and secure, and need more assurances from cloud providers that all measures are being implemented to guarantee just that.

The implementation of this thesis had multiple objectives and contributions that were explained in chapter \ref{cha:introduction} and the main solution proposed was the \textbf{design and implementation of a cloud-enabled privacy-enhanced solution}, with all-in-the-box planned features, and able to be used as a “cloud-platform as a service” solution, providing:

\begin{itemize}
	\item \textbf{Trust-ability, security and privacy}.
	\item \textbf{Software attestation}.
 	\item \textbf{Replication Mechanism}.
  	\item \textbf{Drastically reduced TCB}.
  	\item \textbf{Complete analysis report}. 
\end{itemize}

Given the objectives and contributions proposed, we were able to successfully design and implement a system, leveraging Intel's \gls{SGX} trusted environment execution processor, and provide a trusted and privacy-enhanced in-memory data store, using container technology for the deployment and easy portability, providing remote attestation in order to guarantee software and hardware correctness, drastically reducing the \gls{TCB}, maintaining data secure and private while operating on remote machine.

Limitations of the secure enclave technology were considered, and the use of a dataset operating in unprotected memory but maintaining data private with the help of partial homomorphic encryption with operations performed over encrypted data can be used to overcome some of those limitations. Also, the secure and replicated system was tested and seams viable with a good trade-off between security and performance for some use cases.

\section{Open Issues and Limitations}
\label{sec:open_issues_and_limitations}

During the design and implementation of the system, some issues and limitations appeared. The \gls{SGX} performance and memory limitations are still an issue and, to some use cases, this solution might not be the best suitable solution to adopt. Also, the work with homomorphic encryption is still a on going study case, and it means that this is not a full fledge solution, and may not support all operations over an encrypted dataset.

As any work implemented for a specific platform, this solution will only be effective on Intel machines with a processor that support the latest software guard extensions technology and although the code is easily deployable using containers, the processor is yet not available massively thus, reducing the portability factor. 

One important issue that was found in the implementation of the system is that Redis storage services data persistency, when running on an isolated secure enclave, is exposing data in plain text when trying to persist itself and backing up data into disk. This is a big vulnerability, that was patched temporarily by disabling data persistency on disk, however, data backups are an important feature, and the \textit{SCONE} technology may have a way to resolve the issue with their encrypted file system support, but the problem did not had further investigation.

\section{Future Work Directions}
\label{sec:future_work_directions}

After all the work performed and all some limitations and issues cataloged the following are future work that can be performed to improve a trusted and privacy enhanced in-memory datastore:

The use of the still in progress but already available second version of the secure processor, the \textbf{\gls{SGX}2} will help to overcome some memory limitations and problems by providing dynamic memory swapping and allocation and dynamic enclave resizing at runtime, something that was not possible to test in this dissertation, but will improve not only in enclave memory, but also startup times and the whole system resources consumptions.

The \textbf{proxy server} is identified as a single point of failure, and although no tested in the current project, the replication of the proxy server can help not only with performance, but also with system availability.

Also, although the proxy and the system can support multiple users with multiple roles, it does not offer \textbf{multi-tenant support}, meaning that all data is secured with the same encryption and decryption keys. The usage of different keys per user/tenant, proxy generated or user supplied keys and also a key rotation system can help privacy and improve the overall security of the system.

The implemented system allows for the user to request software and hardware stack information on demand, but the trusted \gls{SGX} attestation procedure, only happens at enclave startup. Some investigation is necessary but, the implementation of an on-demand \gls{SGX} enabled mechanism is very interesting and an important asset to the system.

Mentioning the limitations described in section \ref{sec:open_issues_and_limitations}, fixing the backup issue would be a really important improvement for the system, to allow for data backup and system recovery in the event of a crash of terminal failure.

Finally the system was implemented with the help of the \textit{SCONE} framework, but it would be very important and informative to try and test the same system on another \gls{SGX} or enclave enabled platform and compare the performances, security and resource usage between the different frameworks.

There are always ways to generally improve performance and resource consumption of the infrastructure, and the improvement of a secure and trusted system is a plus for all users that wish to hold control of their data, keep it private and authentic and never relinquish control to anyone, regardless of the environment on which their software runs on.