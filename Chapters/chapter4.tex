%!TEX root = ../template.tex

\chapter{Elaboration Plan}
\label{cha:elaboration_plan}

This chapter proposes a work plan for the elaboration phase of this dissertation. First, in section \ref{sec:workplan}, we summarise the planned tasks as well as their planned time frames. On the annexes it is presented a Gantt chart and a table better detailing the tasks, as well as referring the relevant related dependencies and related milestones.

\section{Work Plan}
\label{sec:workplan}

Our work plan for the elaboration phase will be conducted during the period from 2/Mar/2020 to 15/Sep/2020, summarised as follow:

\begin{itemize}
	\item \textbf{2-13/Mar}: System model and architecture refinements and implementation specifications.
	\item \textbf{9/Mar-8/Apr}: Setup of the development environment and implementation and test of a pilot-prototype for a SGX docker-based  small-scale and single REDIS instance.
	\item \textbf{13/Apr-22/May}: Development of the planned solution (first prototype).
	\item \textbf{4/May-29/May}: Functional, architectural and initial performance evaluation tests with the first prototype.
	\item \textbf{1/Jun-3/Jul}: Development of the planned solution (second prototype, including a replicated solution full-compliant with the primary-backup and cluster model as provided by the original REDIS solution).
	\item \textbf{22/Jun-10/Jul}: Functional, architectural and initial performance evaluation tests with the first prototype.
	\item \textbf{13/Jul-25/Jul}: Deployment of the final Prototype as a Cloud-Based Solution as a Service (running in OVH dedicated instances and datacenters).
	\item \textbf{22/Jul-8/Aug} and \textbf{22/Aug-18/Sep}: Final tests in the Cloud Prototype.
	\item \textbf{30/Mar-18/Sep}: Thesis report writing (in distributed cycles addressing different chapters and the final review phase planned for 11-18 Sep).
\end{itemize}

Figure \ref{fig:workplan_details} details the tasks shown in the work plan Gantt Chart (figure \ref{fig:work_plan}). For each task a description is available to better explain the work planned for the task, and a dependency to another task if relevant.

There are 10 main categories, called milestones or epics, that were identified: \textbf{System Specification}, \textbf{Development Environment}, \textbf{Production Environment}, \textbf{Prototype Setup}, \textbf{Prototype}, \textbf{Prototype Communication}, \textbf{Experimental Evaluation}, \textbf{Prototype Publication}, \textbf{Dissertation Writing} and \textbf{Dissertation Presentation}. Each task will be appointed to epic task.